\documentclass[12pt,a4paper]{article}
\usepackage{tabularx}
\usepackage{booktabs}
\usepackage{longtable}
\usepackage{ltxtable}
\usepackage[latin1]{inputenc}
\usepackage{amssymb}
\usepackage[]{graphicx,rotating}
\usepackage[T1]{fontenc}
\usepackage{parskip}
\usepackage{listings}
\usepackage{natbib}
\usepackage[official]{eurosym}
\usepackage{mathrsfs}
\usepackage{amsmath}
\usepackage{verbatim}
\usepackage[usenames,dvipsnames]{color}     %for R colors and formatting

\usepackage[left=3cm, right=2.5cm, top=2.5cm]{geometry}

\pagestyle{empty}
\parindent 0cm
\renewcommand{\baselinestretch}{1}
\newcommand{\bs}{\boldsymbol}
\renewcommand{\familydefault}{cmss}
\pdfminorversion=7
\bibliographystyle{agsm}

\lstset{ %for R colors and formatting
  language=R,                     % the language of the code
  basicstyle=\scriptsize\ttfamily, % the size of the fonts that are used for the code
  numbers=left,                   % where to put the line-numbers
  numberstyle=\scriptsize\color{Blue},  % the style that is used for the line-numbers
  stepnumber=1,                   % the step between two line-numbers. If it is 1, each line
                                  % will be numbered
  numbersep=5pt,                  % how far the line-numbers are from the code
  backgroundcolor=\color{white},  % choose the background color. You must add \usepackage{color}
  showspaces=false,               % show spaces adding particular underscores
  showstringspaces=false,         % underline spaces within strings
  showtabs=false,                 % show tabs within strings adding particular underscores
  frame=single,                   % adds a frame around the code
  rulecolor=\color{black},        % if not set, the frame-color may be changed on line-breaks within not-black text (e.g. commens (green here))
  tabsize=2,                      % sets default tabsize to 2 spaces
  captionpos=b,                   % sets the caption-position to bottom
  breaklines=true,                % sets automatic line breaking
  breakatwhitespace=false,        % sets if automatic breaks should only happen at whitespace
  keywordstyle=\color{RoyalBlue},      % keyword style
  commentstyle=\color{YellowGreen},   % comment style
  stringstyle=\color{ForestGreen}      % string literal style
} 

\begin{document}

\begin{center}
% \vspace*{1cm}
 \includegraphics[width=0.35\textwidth]{GU-Logo-blau-CMYK.eps} \vspace{2cm}
  
  {\Large{\bf Estimating Market Shares and Optimal Prices with a Heterogenous Multinominal Logit Model}} \medskip

  {\Large{An Empirical Study of TNS Beer Market Data}} \vspace{0.5cm}

  Term Paper \\\vspace{2cm}
  submitted to \\\vspace{0.5cm}
  \textbf{Prof. Dr. Thomas Otter} \\\vspace{0.5cm}
  Goethe University Frankfurt am Main \\
  School of Business and Economics \\
  Chair of Services Marketing \vspace{2cm}
  
  by \\\vspace{0.5cm}
  \textbf{Lukas J\"urgensmeier} \\
  (Mat.-Nr.: 6904281) \\
  
  \bigskip

  in partial fulfillment of the requirements of the lecture \medskip

 {\bf Customer Satisfaction and Consumer Choice} \\
  Summer Semester 2019\\
  \medskip

  and for the degree of \medskip

  \textbf{Master of Science in Business Administration} \\\vspace{0.5cm}
  July 23, 2019
  
\end{center}


\pagebreak
\pagestyle{plain}
\pagenumbering{Roman}
\tableofcontents
\listoffigures
\listoftables
\newpage
\setcounter{page}{2}
\pagenumbering{arabic}
\setlength{\baselineskip}{1.5\baselineskip}
\pagestyle{plain}


\section{Contextual Introduction to the Beer Market}
asd
\section{Motivation of the Analytical Model}

\section{Exogenous Variables and Expected Relationonships}
asd
\section{Model Description and Results}
asd
\section{Market Simulation}
asd
\section{Managerial and Research Implications}
asd
\clearpage
\appendix
\section{R Code}

\begin{lstlisting}[language=R,caption={Estimation Code}, label=lst:estim]
###############################
### Estimation for Beer #######
###############################


## Prepare and load data

rm(list=ls())

library(WriteXLS)
library(Rcpp)
library(devtools)
library(RcppArmadillo)
library(MASS)
library(lattice)
library(Matrix)
library(xtable)
library(bayesm)

set.seed(66)

###Increase memory capacities
memory.limit(size=1000000)

load("Estimation_Data_Beer_20170423.Rdata")

products = c("Amstel Extra Lata 37,5 cl","Amstel Extra Lata 33 cl","Amstel Lata 37,5 cl","Amstel Lata 33 cl","Amstel Cl?sica Lata 33 cl",
             "Cruzcampo Lata 33 cl","Estrella Damm Lata 33 cl","Estrella Galicia Lata 33 cl","Heineken Lata 33 cl","Mahou 5 Estrellas Lata 33 cl",
             "Mahou Cl?sica Lata 33 cl","San Miguel Lata 33 cl","Voll Damm Lata 33 cl","Steinburg (Marca Blanca Mercadona) Lata 33 cl",
             "Marca Blanca Carrefour Lata 33 cl")

N = length(E_Data$lgtdata)

for(i in 1:N){
  colnames(E_Data$lgtdata[[i]]$X) = c(products,"Price")
}

colnames(E_Data$lgtdata[[101]]$X)

save(E_Data,file="Estimation_Data_Beer_20170423.Rdata")


## estimation preparation for bayesm package
Prior = list(ncomp=1)
Mcmc=list(R=6000,keep=2)

out_HB = rhierMnlRwMixture(Data=E_Data,Prior=Prior,Mcmc=Mcmc)
beta_HB = out_HB$betadraw
compdraw_HB = out_HB$nmix$compdraw
probdraw_HB = out_HB$nmix$probdraw

windows()
plot(out_HB$loglike, type="l")

###Get rid of burnin
burnin = 1000
R = dim(beta_HB)[3]

beta_HB = beta_HB[,,(burnin+1):R]
compdraw_HB = compdraw_HB[(burnin+1):R]
probdraw_HB = probdraw_HB[(burnin+1):R]

###EVALUATION
R = dim(beta_HB)[3]
N = dim(beta_HB)[1]

l = 100
index = rep(rank(runif(R)),l)

beta_HP <- array(0,dim=c(R*l,dim(beta_HB)[2]))
#simulate from posterior predictive density of beta (hierarchical prior)
#simulate from posterior predictive density of beta (hierarchical prior)
#simulate from posterior predictive density of beta (hierarchical prior)
#  check ?rmixture
for(j in 1:(R*l)){
  beta_HP[j,] = rmixture(1,probdraw_HB[index[j]],compdraw_HB[[index[j]]])$x    
}


#######resampling from out_HB$betadraw
beta_HP <- array(aperm(beta_HB,perm=c(1,3,2)),dim=c(dim(beta_HB)[1]*dim(beta_HB)[3],dim(beta_HB)[2]))
# beta_HP2 <- NULL
# for(j in 1:R){
#   beta_HP = rbind(beta_HP,beta_HB[,,j])
#   }

#######using posterior means
#######using posterior means
#######using posterior means
beta_HP <- rowMeans(beta_HB,dim=2)


#Illustrate specified distribution graphically
windows()
par(mfrow=c(4,4))
hist(beta_HP[,1], freq = FALSE,breaks=100,xlab="BETA",ylab="DENSITY",
     main=paste("Attribute 1 Level 1:", round(mean(beta_HP[,1]),digits = 2)));grid()
hist(beta_HP[,2], freq = FALSE,breaks=100,xlab="BETA",ylab="DENSITY",
     main=paste("Attribute 1 Level 2:", round(mean(beta_HP[,2]),digits = 2)));grid()
hist(beta_HP[,3], freq = FALSE,breaks=80,xlab="BETA",ylab="DENSITY",
     main=paste("Attribute 2 Level 2:", round(mean(beta_HP[,3]),digits = 2)));grid()

hist(beta_HP[,4], freq = FALSE,breaks=100,xlab="BETA",ylab="DENSITY",
     main=paste("Attribute 1 Level 1:", round(mean(beta_HP[,1]),digits = 2)));grid()
hist(beta_HP[,5], freq = FALSE,breaks=100,xlab="BETA",ylab="DENSITY",
     main=paste("Attribute 1 Level 2:", round(mean(beta_HP[,2]),digits = 2)));grid()
hist(beta_HP[,6], freq = FALSE,breaks=100,xlab="BETA",ylab="DENSITY",
     main=paste("Attribute 2 Level 2:", round(mean(beta_HP[,3]),digits = 2)));grid()

hist(beta_HP[,16], freq = FALSE,breaks=80,xlab="BETA",ylab="DENSITY",
     main=paste("Price:",  round(mean(beta_HP[,16]),digits = 2)));grid()


###Matrix of product combinations (ignoring price)
#comb_m = rbind(c(1,0,1,0),c(0,1,1,0),c(1,0,0,1),c(0,1,0,1))

###Specify grid for price
min_p = 0
max_p = 1 
step = 0.01 #.5
price_grid = seq(min_p,max_p,step)
grid_length = length(price_grid)

# create matrix with two columns: First one only 1 for our product, second one price grid
comb_m_p <- cbind(rep(1,grid_length), price_grid)

###Product Optimization###

###Approximate expected market share 

NUT_agg_esti = beta_HP[,-c(2:15)]%*%t(comb_m_p) # remember identification of the choice likelihood? 
#NUT_agg_esti = beta_HP[runif(dim(beta_HP)[1])>.9,]%*%t(comb_m_p[,-3]) # without (randomized) subsetting the object will be too big for the workspace

Exp_esti = exp(-NUT_agg_esti) 
sc_esti_all = 1/(1+Exp_esti) #Compute share of choice (sc) ~ market share from monopolistic perspective
sc_esti = apply(sc_esti_all,2,mean) #Compute mean over draws to approximate the integral (expected value)
#Compute profits 
costs_A2 = 0.1
profits_esti = array(0,dim=c(length(sc_esti),dim(costs_A2)[1]))
grid_price_minus_cost = price_grid - costs_A2
profits_esti = sc_esti * grid_price_minus_cost

# for(i in 1:dim(costs_A2)[2]){
#   profits_esti[,i] = sc_esti * grid_price_minus_cost[,i]
# }
#Compute optimal product for each possible cost combinations
optimal_product_esti = array(0,dim=c(dim(costs_A2)[1],1))
optimal_product_esti = which(profits_esti == max(profits_esti[]), arr.ind = TRUE)
# for(i in 1:dim(costs_A2)[2]){
#   optimal_product_esti[i] = which(profits_esti[,i] == max(profits_esti[,i]), arr.ind = TRUE)
# }

###Plot profit curve for each cost scenarion
plot(profits_esti[],col = "red",type="l",xlab="Price", main="Optimal price of first brand in monopolistic market" ,ylab="Profits");grid()
abline(v=optimal_product_esti[],col = "red",lty=3,lwd=3)


windows()
par(mfrow=c(3,1))  # multiple plots are filled by rows!!!
plot(profits_esti[,1],col = "red",type="l",xlab="Product Index", main="Cost Scenario I" ,ylab="Profits");grid()
abline(v=optimal_product_esti[1],col = "red",lty=3,lwd=3)
legend("bottomleft",expression("Profits Estimated"),cex=1,lty=c(1),lwd=c(1),col=c("red"))
plot(profits_esti[,2],col = "red",type="l",xlab="Product Index", main="Cost Scenario II" ,ylab="Profits");grid()
abline(v=optimal_product_esti[2],col = "red",lty=3,lwd=3)
legend("bottomleft",expression("Profits Estimated"),cex=1,lty=c(1),lwd=c(1),col=c("red"))
plot(profits_esti[,3],type="l",col = "red",xlab="Product Index", main="Cost Scenario III" ,ylab="Profits");grid()
abline(v=optimal_product_esti[3],col = "red",lty=3,lwd=3)
legend("bottomleft",expression("Profits Estimated"),cex=1,lty=c(1),lwd=c(1),col=c("red"))

###Summarize results in one matrix
Optimal_Product_ALL = NULL
Optimal_Product_esti <- array(0,dim=c(3,6))
rownames(Optimal_Product_esti) = c("Scenario I","Scenario II","Scenario III")
colnames(Optimal_Product_esti) = c("Attribute 1 Level 1", "Attribute 1 Level 2", "Attribute 2 Level 1", "Attribute 2 Level 2", "Price", "Profits")
for(i in 1:dim(costs_A2)[2]){
  Optimal_Product_esti[i,] = c(comb_m_p[optimal_product_esti[i],],profits_esti[optimal_product_esti[i],i])
}

###Optimal products from estimation
Optimal_Product_esti
###True Optimal products
Optimal_Product

\end{lstlisting}


\clearpage
%\bibliography{library}

\newpage
\thispagestyle{empty}
\section*{Statutory Declaration}

I herewith declare that I have completed the present term paper independently, without making use of
other than the specified literature and aids. Sentences or parts of sentences quoted literally are
marked as quotations; identification of other references with regard to the statement and scope of
the work is quoted. The thesis in this form or in any other form has not been submitted to an examination body and has not been published.
This thesis has not been used, either in whole or part, for another examination achievement.

\vspace{1cm}

Frankfurt am Main, July 23, 2019
\vspace{2cm}

. . . . . . . . . . . . . . . . . . . . . . . . . . . . . . .
\vspace{0.1cm}

Lukas J\"urgensmeier
\end{document}
